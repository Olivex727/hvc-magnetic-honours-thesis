\chapter{Magnetic Field Derivation}
\label{cha:derivation}

In constructing a rudimentary algorithm to quantify the effect of magnetic draping on HVCs, as in section \ref{sec:outline}, it is important to pay attention to the validity of the methods being employed and less the accuracy. As an inaccurate methodology can be improved using better data, removing assumptions, and more precision, however an invalid methodology cannot provide a proper estimate whatsoever.

\section{Assumptions}
\label{sec:assumptions}

Due to the significant limitations in available data, the use of assumptions is necessary to begin the process of rough evaluation. Several assumptions were employed in the process of evaluation, some of which are less than satisfactory at contributing an accurate answer to the research question. Again, it is highly important to note that goal is to ensure the validity of the method is sound and to ensure the method's manipulability, but not necessarily provide a guarantee of accuracy and precision.


The first major assumption is that the ionisation fraction of the Galactic halo is approximately constant, i.e. gas in the halo is well-mixed. This does not account for major sources of heterogeneity caused by HVCs (a desired outcome) or other ionised regions such as the Magellanic Clouds (an undesired outcome). It is not necessarily unreasonable to assume this, especially if the region of analysis per HVC is under 4 square degrees.


Simulation data can be used to analyse the problematic nature of assuming a constant ionisation fraction \citep{ID23}. However, it is difficult to apply simulation data to observational data due to the sheer variety of analysis capable of being performed by simulations. For now, it is 'good enough' to rely on that assumption from \cite{ID23} that the ionisation fraction is high and constant.


The second major assumption is that the weighted average of the HI column density surrounding a HVC is relatively constant and biased towards the peak column density in the centre of the HVC. This is an assumption that only works under the consideration of averages, as HI column density is certainly not uniform within clouds, as mentioned in \cite{ID69}.


Both above assumptions can, and should, be taken to be unreasonable for providing an accurate estimate of magnetic field strength. However, the result of applying both assumptions is that the line-of-sight magnetic field strength is linearly proportional to the faraday depth with a scaling constant unique to each HVC – as demonstrated in later section \ref{sec:los_dev}. This greatly simplifies the analysis, without loss of validity, primarily due to the method of calculation explained by \cite{ID27} and in section \ref{sec:los_dev}.


The third core assumption is that HVCs will appear as approximately circular within the field. Meaning that to distinguish between the surrounding medium and HVC RM grids, a simple circular method can be applied. As seen from the HI plots in figure \ref{fig:all_hvcs}, this assumption appears to only be correct in particular cases. But do note that the HI background discriminates for high-VLSR HI emission.


All three assumptions additionally meet the desired goal of an alterable method. As discussed in the discussion and conclusions sections \ref{ssec:B2} and \ref{ssec:mag_conclusion}, these three assumptions can be dropped in the presence of more detailed data or more rigorous methodologies manipulated from the one described in this paper.


Lastly, a very typical assumption made is that the RM contribution along a line of sight is averaged across said line of sight. This is not necessarily correct; however, it is a fair assumption that most past literature has made to deconstruct the integral as seen in \cite{ID27, ID3, ID26}. This makes the fourth assumption.


\section{Derivation of Line-of-Sight Magnetic Fields}
\label{sec:los_dev}

As from equation \ref{eq:rm_integral}, there is a relationship between the line-of-sight magnetic field and the faraday depth. By removing foreground contributions to the faraday depth, one can isolate the specific contribution of the line-of-sight magnetic field made by the HVC. At the same time, the fourth assumption can be used to solve evaluate the integral, resulting as follows \citep{ID26}:


\begin{equation}
    \frac{\left<B_{\parallel}\right>}{\mu\mathrm{G}}=\frac{\left<\Phi_{\mathrm{cor}}\right>}{0.81\left<n_e\right>L_{H^+}}
\label{eq:B_intermediate}
\end{equation}


Where $\left<B_{\parallel}\right>$ is the average line-of-sight magnetic field in \textmu G, $\Phi_{\mathrm{cor}}$ is the corrected faraday depth in $\mathrm{rad m}^{-2}$, $n_e$ is the electron density in $\mathrm{cm}^{-3}$, and $L_{H^+}$ is the HVC path length in pc. From this equation, \cite{ID27} derives a simplified equation that eliminates the need to calculate the electron density and path length:


\begin{equation}
    \frac{B_{\parallel}}{\mu\mathrm{G}}=3.80\times10^{18}\frac{\Phi_{\mathrm{cor}}}{X\left<N_{HI}\right>}
\label{eq:the_equation}
\end{equation}


Where $X$ is the unitless ionisation fraction, and $\left<N_{HI}\right>$ is the weighted average HI column density over the region being analysed, measured in $\mathrm{cm}^{-2}$. This equation can only be derived under the first assumption. Additionally, the use of the average HI column density is intrinsically tied to the second assumption.


Applying both assumptions and utilizing this equation demonstrates that the line-of-sight magnetic field is linearly proportional to the faraday depth. For every RM sampling point, the magnetic field was calculated using this equation. With the assumption that the ionisation fraction is approximately unity, and the average HI column density is equal to the peak HI column density of the HVC. As aforementioned, these values act more as simple placeholders for more detailed analysis, being within an order of magnitude is all that is necessary for the aims of this report.


At this point, it is important to distinguish between what is called 'virtual magnetic field' and 'actual magnetic field'. For any individual RM point source, there are a lot of interferences that may make the resulting magnetic field not representative of the actual magnetic field at that point – more specifically, the turbulent nature of gas mixing in the halo and HVC \citep{ID69, ID30, ID16}. Thus, calculating the magnetic field value associated with a single RM point source is to be considered 'virtual' – an approximation to the actual magnetic field within a degree of uncertainty. It is only through the statistical analysis of multiple virtual magnetic fields that an actual magnetic field value can be derived. As the combination ensures that turbulence effects are accounted for, due to it being unlikely that random contributions to line-of-sight magnetic fields can become uniform at the point of observation.


Despite the detailed analysis conducted in section \ref{cha:FR}, it was decided that the non-altered interpolated sky map be used to correct RM contributions. This is due to most already-existing literature relying on this method. From figure \ref{fig:anisotropy}, which displays histograms of the corrected and uncorrected line-of-sight magnetic field points surrounding HVCs. From the figure, it is demonstrated that the interpolation can remove the anisotropy of the ISM by turning an uncorrected bimodal symmetric distribution into a normal distribution centred at zero. However, it also is apparent that the correction can introduce a significantly higher variance in the data, shown in the ridge-like structures in figure \ref{fig:all_hvcs}.

\begin{figure}
    \includegraphics[width=10cm]{"Virtual_B_All_HVCs.png"}
    \centering
    \caption{Two histograms describing the calculated line-of-sight virtual magnetic field strengths within a certain distance to the 15 sample HVCs. The red histogram uses uncorrected. The green histogram uses corrected RMs.}
    \label{fig:anisotropy}
\end{figure}


\section{KS Testing}
\label{sec:KStest}

Before evaluating the effect of magnetic draping on HVCs, it is important to first confirm that there is a detectable magnetic field to analyse. To do this, the KS test was employed. The set of RM grid points for each HVC collated in section \ref{ssec:hvc_snapshot} include any grid point within $2 - 2\pi$ degrees of the HVC centre, depending on HVC size. Employing assumption three, two separate populations can be formed by delineating RMs inside and outside a certain circular distance from the HVC centre. The specific radius of delineation was the average value between the Moss catalogue's 'dx' and 'dy' values (i.e. its cartesian dimensions). All sampling points within the defined circle is apart of the “HVC RMs” population, and the ones outside the circle are apart of the “Background RMs” population.


Due to the linear proportionality between the faraday depth and virtual magnetic field, these two values are interchangeable when doing statistical analysis on the individual HVC level. This allows the KS test to be performed on the set of virtual magnetic fields associated with grid points directly, instead of the faraday depth.


A two-sample, two-tailed, KS test was performed comparing both the background and HVC virtual magnetic fields. A critical p-value of 0.001 was assumed to determine if a significant magnetic field was detected. Table \ref{tab:KStest} shows the KS test results applied to each HVC. Out of the 15 HVCs tested, 11 HVCs (73\%) had a significant difference between both populations. This is a promising confirmation of the hypothesis presented by the literature, that magnetic draping is an existing phenomenon.

\begin{table}
    \centering
    \begin{tabular}{l l l}
        \hline
        \bfseries Name & \bfseries KS Statistic & \bfseries p-value \\
        \hline
        \csvreader[head to column names]{"../../Resources/CSV/KStest_proc.csv"}{}
        {\\\csvcoli & \csvcolii & \csvcoliii}
    \end{tabular}
    \caption{A table describing the KS test results for each HVC.}
    \label{tab:KStest}
\end{table}


\section{Mathematical methods to evaluate HVC Magnetic Fields}
\label{sec:evaluation}

For the HVCs which did have detectable magnetic fields, the next step is to estimate the strength of the magnetic field draping over each HVC. Previous analyses of the Smith Cloud, such as in \cite{ID5, ID26}, utilises a weighted average to determine the magnetic field strength. However, this method is not transferable to the generalised algorithm. In \cite{ID5, ID26}, due to analysing one single and large object, it is possible to divide the object into several smaller sections for analysis. It is not possible to do this especially given the limitations created by assumption three.


While the background virtual magnetic field population will generally follow the expected normal distribution, the in-HVC population can come in multiple varieties including: skewed normal distributions, where the HVC is travelling side-on; bimodal distributions, where the HVC is travelling towards or away from Earth; or sinc distributions, similar to the bimodal distribution, however under the unique case where the HVC magnetic field perfectly overlaps with the corrected background, resulting in a depopulation of virtual magnetic fields where the HVC magnetic field strength should align to. More is provided on these morphologies in section \ref{sec:toy_models}.


The more complex HVC virtual magnetic field distribution combined with the oversimplified morphology model means that using a weighted average is dubious in validity at best. Henceforth, this report proposes two new methods to determine the magnetic field strength surrounding HVCs under the simplified morphological model. Sections \ref{ssec:KS_EDF} and \ref{ssec:sigma_sub} attempts to justify the mathematical validity of these two methods, section \ref{sec:toy_models} demonstrates validity further through the application of a simple toy model, and section \ref{sec:results} shows both the results of using these methods, with some further post-hoc justifications demonstrated in output data.


\subsection{KS-EDF Method}
\label{ssec:KS_EDF}

E

\subsection{Uncertainy Subtraction}
\label{ssec:sigma_sub}

E

\section{Toy Model Analysis}
\label{sec:toy_models}

E

\section{Magnetic Field Results}
\label{sec:results}

E

\subsection{Uncertainties}
\label{ssec:results_uncertainties}

E


%%% Local Variables: 
%%% mode: latex
%%% TeX-master: "paper"
%%% End: 
