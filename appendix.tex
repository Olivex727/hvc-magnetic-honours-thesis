\setcounter{chapter}{7}
\setcounter{section}{0}

\renewcommand*{\thechapter}{}

\appendix

\chapter{Appendix}
\label{cha:appendix}

\renewcommand*{\thesection}{\Alph{section}}

\section{Developed Code and Data}
\label{sec:appendixA}

All developed code and data is found in a publically-available GitHub repository as shown here: \url{https://github.com/Olivex727/hvc-magnetic-honours-programs}.

\section{All HVCs}
\label{sec:appendixB}

The filtered Moss catalogue is displayed in table \ref{tab:moss_hvcs}. The table has been modified from what was originally retrieved for display and formatting purposes. The actual Moss catalogue data can be found at VisieR: \url{https://vizier.cds.unistra.fr/viz-bin/VizieR-3?-source=J/ApJS/209/12}.

\begin{table}
\renewcommand\thetable{A}
\begin{flushleft}
\begin{tabular}{l | l l l l l l}
    \hline
    Name & RAJ200 & DEJ2000 & VLSR & VLSR Error & VGSR & V Dev. \\
    \hline
    \csvreader[head to column names]{"./csv/moss_filtered.csv"}{}
    {\\ \csvcoli & \csvcolii & \csvcoliii & \csvcoliv & \csvcolv & \csvcolvi & \csvcolvii}
    \\
    \hline
\end{tabular}

\begin{tabular}{l | l l l l l l l}
    \hline
    Name & FWHM & $T_b$ Fit & $N_{HI}$ & $N_{HI}$ Error & Area & dx & dy \\
    \hline
    \csvreader[head to column names]{"./csv/moss_filtered.csv"}{}
    {\\ \csvcoli & \csvcolviii & \csvcolix & \csvcolx & \csvcolxi & \csvcolxii & \csvcolxiii & \csvcolxiv }
    \\
    \hline
\end{tabular}
\end{flushleft}
\caption{Numerical values for all 13 HVCs being analysed in the report. Retrieved from \cite{ID3}.}
\label{tab:moss_hvcs}
\end{table}

\section{Planck Mission Cosmic Microwave Background Images}
\label{sec:appendixC}

The source for the Planck mission's CMB temperature map is available on this website: \url{https://www.esa.int/ESA_Multimedia/Images/2013/03/Planck_CMB/}.

\section{PyNUFFT Python Module}
\label{sec:appendixD}

The pyNUFFT Python Module, which was used in the investigation for foreground removal, has its main documentation page here: \url{https://pynufft.readthedocs.io/en/latest/index.html}

\newpage

\section{Statistical ANOVA Tests}
\label{sec:appendixE}

The R language was used to calculate all tests mentioned in \ref{ssec:results_stats}, the file is available here: \url{https://github.com/Olivex727/hvc-magnetic-honours-programs/blob/main/KS_confirmation/wgt_anova.R}. This R-Code, in addition to the research of the Weighted ANOVA and Tukey HSD test was aided by the ANU SSN Consultancy Group. See the Acknowledgements for more.

The specific R code output when running this file is as follows:

\begin{verbatim}
> summary(base_V2_aov)
Df Sum Sq Mean Sq F value Pr(>F)  
data_new$variable.x  2   44.7  22.349   3.375 0.0388 *
Residuals           87  576.1   6.622                 
---
Signif. codes:  0 '***' 0.001 '**' 0.01 '*' 0.05 '.' 0.1 ' ' 1
> tukey.test
Tukey multiple comparisons of means
95% family-wise confidence level

Fit: aov(formula = data_new$Estimate ~ data_new$variable.x,
weights = data_new$prescision)

$`data_new$variable.x`
   diff       lwr         upr     p adj
Var_Sub-KS_EDF   -1.199 -2.783335  0.38533482 0.1741072
Wgt_Mean-KS_EDF  -1.675 -3.259335 -0.09066518 0.0357301
Wgt_Mean-Var_Sub -0.476 -2.060335  1.10833482 0.7544765

> 
\end{verbatim}

A copy of this text output is available in the file: \url{https://github.com/Olivex727/hvc-magnetic-honours-programs/blob/main/KS_confirmation/chisq.txt}.

%\begin{figure}
%    \renewcommand\thefigure{A}
%    \includegraphics[width=10cm]{"Lockman_2008.jpeg"}
%    \centering
%    \caption{From \cite{ID28}, figure 1. A HI image of the Smith Cloud taken from the Green Bank Telescope at a local standard of velocity rest of 100 $\mathrm{kms^{-1}}$. The purpose of the arrows are not meaningful to this paper.}
%    \label{fig:sc}
%\end{figure}