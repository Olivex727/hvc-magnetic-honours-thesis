\chapter{Conclusions}
\label{cha:conclusion}

Of 13 HVCs analysed in this report, a detection of magnetic draping of 11 HVCs was found to within 99.9\% confidence. Two new methods (KS-EDF, and variance subtraction) to evaluate the magnitude of magnetic draping were proposed, in addition to the use of a previously sanctioned method (weighted average). All three methods agree with each other to within 95\% confidence. Despite this, the KS-EDF method does not appear to be a viable method for analysis. Whereas both the variance subtraction and weighted mean methods agree with each other, having a reduced chi-squared statistic of 1.088. The mean and median HVC magnetic field is on the order of magnitude of 1 \textmu G – a factor of 10 larger than in provided simulations (\citeauthor[][simulations]{ID23}). Further research needs to be done in the gathering of ionisation fraction data, like in \cite{ID67}, improving the accuracy of both methods, and modelling to further validate both methods.

A secondary focus was on the investigation of image processing to solve anticipated issues with higher-quality SKA-era interpolated rotation measure data. It was found that a two-dimensional bandpass filter was effective at preserving the rotation measure distribution of a lower-quality interpolations, while filtering out for objects of desired spatial wavelength. It is important, however, that future investigation is done on different bandpass window shapes and opacities, to reduce rippling effects. While the use of NUFFTs were shown to be deeply ineffective as a combined interpolation and filtering method, it may be more useful when the POSSUM dataset is in its late stages of compilation.

The implications of the magnetic draping hypothesis have significant ramifications on the future of galactic archaeological research. It would necessarily mean that HVCs are protected by magnetic fields, which helps to confirm a lot of past research done on modelling and predictions of HVC survivability (\citeauthor[][simulations]{ID23}). It additionally aligns with other observations of HVCs, specifically the Smith Cloud \citep{ID5, ID26}. While it confirms the efforts made by past research, it can help bring the field one step closer to understanding the origins of pristine gas for star-forming galaxies, with a proven method for how galaxies can obtain said gas. A small proportion of HVCs that did not have significant magnetic field detections – specifically HVCs G089.0-64.7-311 and G248.9+36.8+181. If this were not caused by methodological error, a re-examination of both HVCs is required.

