\chapter{Conclusions}
\label{cha:conclusion}

Of 13 HVCs analysed in this report, a statistical detection of a magnetic field signature of 11 HVCs was found to within 99.9\% confidence. Two new methods to evaluate the magnitude of magnetic fields were proposed (KS-EDF, and variance subtraction), in addition to the use of a previously sanctioned method (weighted average). All three methods agree with each other to within 95\% confidence. Despite this, the KS-EDF poses more problems for uncertainty and estimation, at least statistically. Whereas both the variance subtraction and weighted mean methods agree with each other, having a reduced chi-squared statistic of 1.088. The mean and median HVC magnetic field is on the order of magnitude of 1 \textmu G – a factor of 10 larger than in some of the provided simulations (\citeauthor{ID23} simulations), but reasonable in the simulations by \cite{ID34}. There are very large uncertainties in all the calculation methods, so conclusions about their efficacy need to be taken with caution.


The investigation focused on a small sample size of HVCs, with a very specific Galactic latitude range - while this is neccesary for preliminary analysis, repeating the estimation but with more HVCs and more data is needed to assess the efficacy of this report's conclusion. The circular HVC assumption is an approximation that should be improved on by a more complex morphology analysis algorithim. Perhaps the use of artificial intellegence, or dimensional analysis, can provide a more definitive deliniation of HVC shape. HI and H-alpha maps can be used to additionally remove the shortfalls of assuming a singular HI column density for each HVC. Further research needs to be done in the gathering of ionisation fraction data, like in \cite{ID67}. A more accurate estimate of ionisation fraction can help improve the reliability of the magnetic field estimates. Modelling should be conducted to validate the methods of estimation discussed in this report.


A secondary focus was on the investigation of image processing to solve anticipated issues with higher-quality SKA-era interpolated rotation measure data. It was found that a two-dimensional bandpass filter was effective at preserving the rotation measure distribution of a lower-quality interpolations. The next step is applying FTs to high-definition interpolations to test their efficacy at removing objects of interest from the foreground. Future investigation should also be done on different Crosshatch-Bandpass window shapes and opacities, to reduce rippling effects. While the use of NUFFTs were shown to be ineffective as a combined interpolation and filtering method, it may be more useful when the POSSUM dataset is larger.
