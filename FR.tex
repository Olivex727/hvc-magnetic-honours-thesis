\chapter{Foreground Subtraction}
\label{cha:FR}

Foreground subtraction remains an open question in the field of RM radio astronomy. Due to the inherent magnetisation of the Galactic halo and the ISM, as discussed in sections \ref{ssec:draping} and \ref{ssec:faraday}, there are significant contributions to observed RMs from the Galactic foreground across the entire sky. To correctly determine the magnetic field surrounding objects of interest, one must first remove this source of systematic error.


A prime example of the consequences of not correctly accounting for foreground RM contributions is the paper by \cite{ID2}, which attempted to estimate the magnetic field strength surrounding HVCs in the Leading Arm. However, as from \cite{ID36}, this result is possibly erroneous due to the obstruction of the nearby Antila supernova remnant region. While the analysis of multiple HVCs is more likely to prevent these errors from compounding to invalidate the conclusions of the report, it is still useful to account for these contributions as much as possible.

\section{Foreground Reconstruction via Interpolation}
\label{sec:intp}

All previous work on magnetic field analysis of HVCs, and more broady on radio objects of interest, involve the use of interpolation \citep{ID3, ID5, ID6, ID26, ID73}. Interpolations are beneficial due to their ability to convert a discrete distribution of RM grid points into a continuous distribution of the RM sky. Interpolations also benefit from a 'smoothing' effect; that interpolated maps can smooth out small-scale imperfections in the RM grid that may not correspond to actual foreground objects \citep{ID44, ID45, ID58}. This smoothing can occur because of the interpolation algorithm itself, or the lack of RM grid density to resolve objects on a particular scale.


The primary issue with interpolation is that they are too effective a technique at reconstructing the RM foreground. Despite the smoothing effects that they can provide, there is no way one can confirm that the RM sky has not included objects of interest as well; an interpolated RM sky could contain the profiles of objects of interest. This has the effect of making interpolated maps sourced from high-density RM grids innapropriate for foreground correction, as the original intention of interpolation is to subtract out foreground contributors to RMs – not the objects of interest themselves.


Thus, with the increased RM grid density afforded by the POSSUM survey, and future SKA-era projects, it is of high importance that discussions on how to account for these issues can be solved.

\section{Annulus Subtraction}
\label{sec:annulus}

The immediate alternative is annulus subtraction. This is a method employed across all fields of astronomy, including radio astronomy, being most applicable to single-object analysis. The method generally involves selecting a series of RM sampling points surrounding any given central RM grid point, averaging the selected RMs, and subtracting the average from the central RM grid point.

There are two sub-methods to consider when performing annulus subtraction: fixed-size annulus subtraction and fixed-sampling annulus subtraction. Fixed-size annulus subtraction involves defining an annulus with a constant inner and outer radius and averaging the RMs exclusively within this radial range. Fixed-sampling annulus subtraction involves defining an inner radius and then selecting a fixed amount of RM grid points that are closest to the central point, but still outside the inner radius. Assuming a constant grid density everywhere in the field, a relationship between the two methods can be quantified, as in the following equation:

\begin{equation}
    \begin{split}
        R = \sqrt{r^2 + \frac{N}{\pi n}} \\
        N = \pi n \left( R^2 - r^2 \right)
    \end{split}
    \label{eq:annulus}
\end{equation}

Where n is the RM grid density in $\mathrm{deg}^{-2}$, r is the inner radius of the annulus in degrees, R is the outer radius of the annulus in degrees (which is directly fixed under the fixed-size regime), and N is the number of RM grid points used (which is directly fixed under the fixed-sampling regime) and is unitless. This means that under a constant RM grid density, these two methods should be approximately equivalent.


There are benefits to both methods. On one hand, fixed-size methods can be described mathematically as convolutions, making them linear. However, they can run into measurement and calculation errors when there is a low amount of RM grid points surrounding the central RM. On the other hand, fixed-sampling methods guarantee a consistent uncertainty and the existence of an average. However, this method is both non-linear and prone to including RM grid points very far away from the central point.


The primary issue with annulus subtraction is determining the size of the annulus, or the amount of RM sampling points to select i.e. what counts as the foreground versus the object, what is an effective smoothing of the foreground model? This is what leads the above-mentioned errors in both methods. The logical response is either to select a large-area annulus that completely removes the objects' RM contributions (in the case of the sample HVCs this would correspond to an annulus of $1-\pi$ square degrees in area), or to select a small radius with numerous RM points to capture the foreground contributions both overlapping the object and isolated in the field. Both methods will be analysed in this paper, with the former being discussed in section \ref{ssec:kernel} and the latter being shown in figure \ref{fig:annulus_sky} \footnote{The fixed-sampling annulus subtraction was proprietary data collected from the supervisor. However, the quantitative and qualitative analysis of the viability of this method is my own work.}.

The specific choice of parameters in the fixed-sampling regime being an inner radius of 0.4” and a sample size of 50 grid points – corresponding to an outer radius of approx. $0.728^{\circ}$. These values were selected to test for small-annulus corrections to RM grid points.

\begin{figure}
    \includegraphics[width=12cm]{"POSSUM_RMs_Aitoff_Craig_Annulus.jpg"}
    \centering
    \caption{An Aitoff projection of all portions of the RM sky observed by ASKAP for the POSSUM survey similar to \ref{fig:rm_map}, filtered through a proprietary fixed-sampling algorithim with inner radius of 0.4" and a samplinig constant of 50.}
    \label{fig:annulus_sky}
\end{figure}

\section{Fast Fourier Transforms (FFTs)}
\label{sec:ffts}

Many of the methods for foreground subtraction beyond interpolation have a common intersection point in the form of image-based signal processing. Thus, the introduction of Fourier Transforms (FTs) may be a very useful direction for analysis, as they are the foundation for most signal processing methods \citep{ID38}. The benefit of FTs is their linearity, which in several ways can reduce computational expense: the trivialisation uncertainty calculations (quantified in equation \ref{eq:ft_unc}); the linear combination of several kernel techniques; signal processing in separate orthogonal dimensions; and consistent scaling relationships. FTs also can utilise both convolutional blurring and bandpassing separately, with convolutions already being discussed with annulus subtractions.


FFTs extend the benefits of FTs by providing a FT algorithm of $O(N\mathrm{log}N)$ complexity and allowing FTs to be performed over discrete sets of data. This allows for the analysis of high-definition pixellated images, which is not unlike the standard format and use-case of a FITS file, especially when using a cartesian projection of the sky. Thus, by applying 2-dimensional FFTs to interpolated RM sky images, it is possible to solve the problem introduced by interpolated high-density RM grids.

\subsection{Non-Uniform Fast Fourier Transforms (NUFFTs)}
\label{ssec:nuffts}

FFTs can further be extended to the analysis of non-uniform data sets. Standard FFTs rely on the assumption that the grid of sampling points is uniform, and resultantly output uniform-density frequency distributions. NUFFTs do not require the assumption of uniformity, nor do they need to output uniform-density frequency distributions \citep{ID55, ID57}. This means that instead of relying on interpolations at all, FTs can be applied directly to the RM grid itself. The primary sources for NUFFTs used in this paper are \cite{ID55, ID57}, with heavy reliance on the python module \verb|PyNUFFT| (see appendix \ref{sec:appendixD} for more).


There are three types of NUFFT: Forward, Adjoint, and "True" \footnote{There is no generally accepted nomenclature for type 3 NUFFTs that align with the single-adjective terminology. So "True" is used because it is an accurate descriptor for the type, involving both non-uniform x and k spaces.}. The forward and adjoint types are inverses of each other – forward NUFFTs take a uniform image and a set of sampling points and return a non-uniform frequency distribution and adjoint NUFFTs reverses that process. True NUFFTs take a non-uniform distribution and output a non-uniform frequency distribution. True NUFFTs are not generally useful for the purposes of this report.


Applying a forward and then an adjoint NUFFT to a set of RM grid sampling points should perform the same task as creating an interpolation. From there, the intermediary step of a bandpass or kernel can be applied to the frequency distribution to produce an interpolation with objects of a particular scale removed from the field.


The important first step in determining if this method can create a reliable interpolation on its own. First, an image was selected, specifically a grayscale and cropped image of the Cosmic Microwave Background (CMB) from the Planck mission (see appendix \ref{sec:appendixC} for more). This was chosen as the test image due to the CMB being able to replicate a noisy and 'blobby' structure, the CMB has also been analysed using FFTs for unrelated cosmological purposes.


Then, a random set of sampling points were selected and treated as the 'mock RMs', with the colour of the background corresponding to the intensity of the RM at that point. The image and the sampling points were then given to the \verb|PyNUFFT| module and transformed in and out of the frequency domain. Figure \ref{fig:nufft} represents the outcomes of this analysis, performed on a sample of simulated RM grids with size $30^{\circ}\times 30^{\circ}$. Ignoring the grid-like structure in the recreated image (a consequence of the random point generation algorithm), even with a very high sampling point density or large field, the image is still very low-quality.

\begin{figure}
    \includegraphics[width=12cm]{"NUFFT_Compare.png"}
    \centering
    \caption{(Right) A cropped and grayscale image of the CMB. (Left) The same image after being fed through a forward then adjoint NUFFT. The amount of sample points total to 27000. Both plots share a common intensity colourbar for refrence.}
    \label{fig:nufft}
\end{figure}


This does not disqualify the NUFFT as an analysis technique. Instead, it means that this technique can only work on a very large continuously connected RM set i.e. a complete or partially complete RM grid map of the sky. However, due to the lack of POSSUM data in its early stages, this is a method that must be investigated in the future. Hence its in-depth inclusion in this report despite its apparent shortcomings.

\subsection{Bandpass Filtering}
\label{ssec:bandpass}

A simpler method is to directly alter the Hutschenreuter map itself using normal FFTs. First, a 2-D FFT was applied to the Hutschenreuter map. A crosshatch-shaped bandpass was created. This crosshatch imitates a bandpass commonly applied to 1-dimensional signals, where regions of a particular angular area are eliminated by removing all frequencies corresponding to that angular area in the k-space. The equation below quantifies the relationship between frequency and angular area:


\begin{equation}
        k_{\mathrm{HVC}} = \frac{1}{2 \theta_{\mathrm{HVC}} R}
    \label{eq:freq_to_angle}
\end{equation}


Where $k_{\mathrm{HVC}}$ is the spatial frequency in $\mathrm{deg}^{-1}$, $\theta_{\mathrm{HVC}}$ is the angular size of the HVC in degrees, and R is the pixel resolution of the axis, in pixels per degree. Assuming all RM sky images exist in a 2:1 cartesian space, due to the range of Galactic latitude and longitude, the value of R is constant across the two axes. The HVC angular size chosen was $1 - \pi$ deg.


The crosshatch is shaped such that, when multiplied by the original k-space, objects of a particular size are either eliminated or reduced in prevalence. This method also guarantees the linearity of the crosshatch 'function'. After this, the inverse FFT is applied to give a resulting foreground map, seen in figure \ref{fig:ripples}.

\begin{figure}
    \includegraphics[width=\textwidth]{"Crosshatch_Interpolation_Compare.png"}
    \centering
    \caption{Crosshatch-Bandpassed versions of the interpolated RM sky at various opacity gradings - where "original" means 0\% opacity. The term "opacity" refers to the effect of the bandpass e.g. 100\% opacity means the bandpass completely removes selected frequencies and 50\% opacity means that the bandpass halves the presence of selected frequencies.}
    \label{fig:ripples}
\end{figure}


However, bandpassing introduces ripples into the Hutschenreuter map. This effect is expected but undesirable. There are two methods to remove this: either to apply the crosshatch at a certain 'opacity' i.e. the crosshatch is not eliminating all the k-space in its region, but instead is reducing those frequencies by a percentage; or using a more complex window than a Top Hat, such as a Tukey window or Gaussian window. The effects of the former are seen in figure \ref{fig:ripples}. The latter was not investigated due to time constraints.


When applying FTs to any interpolated image, it is important to maintain the corresponding uncertainty map's accuracy. This is where one can take advantage of linearity. The following formula below determines how uncertainties can be calculated:


\begin{equation}
    \sigma_{\mathrm{output}} = \mathbf{\mathfrak{F}}^{-1} \biggl[ B \bigl( \mathbf{\mathfrak{F}} \left[ \sigma_{\mathrm{original}} \right] \bigr) \biggr]
    \label{eq:ft_unc}
\end{equation}


Where $B\colon\sigma\rightarrow\sigma$ is the bandpass function, $\sigma_{\mathrm{output}}$ and $\sigma_{\mathrm{original}}$ is the uncertainty images for the output and input respectively in $\mathrm{rad m}^{-2}$, and $\mathbf{\mathfrak{F}}$ is the FT.

\subsection{Kernel Filtering}
\label{ssec:kernel}

The same techniques from above can be applied via convolutions, where the aim is to convolve the Hutschenreuter map with a defined kernel. Two-dimensional convolutions have a time complexity of $O(n^4)$, depending on the kernel size, whereas the FFT has a complexity $O(n^2 \mathrm{log}^2 n)$. By performing a FFT on both the kernel and the Hutschenreuter map separately, then multiplying the two k-spaces together, and applying an inverse FFT, the result is a faster application of a convolution with a kernel. This was the chosen method to demonstrate the large fixed-size annulus subtraction method, with an inner radius of 1 degree and an outer radius of $\pi$ degrees Figure \ref{fig:annulus_interpolation} displays the results of this method.

\begin{figure}
    \includegraphics[width=12cm]{"Annulus_Interpolation_Compare.png"}
    \centering
    \caption{Cartesian plots of the interpolated RM sky (Top) compared against the Annulus-Bandpassed version of the interpolated RM sky (Bottom). The annulus kernel used sums to unity, making it act like a unitary operator, standard for blurring techniques \citep{ID38}. The annulus size ranged from 1-$\pi$ degrees.}
    \label{fig:annulus_interpolation}
\end{figure}

The main problem is that the annulus kernel appears to blur out smaller-scale structures. This can lead to more minute RM changes smaller than the angular size of the HVC may be removed out when it may be neccesary to keep them.

\section{Characterising Methods to Improve Interpolations}
\label{sec:charm}

As seen from figures \ref{fig:snr} and \ref{fig:rm_scatter}, there are several ways in which the RM grid can have 'bad data' – most notably in a lack of signal-to-noise and the inherent scatter when observing near the Galactic midplane. Thus, the final step of this chapter is to both characterise the RM sample set and to compare the subtraction methods against each other.


Figure \ref{fig:big_hist} represents a simple residual histogram comparison between all the methods discussed in this chapter. The desired result is seen in the residuals between the interpolated or crosshatch-bandpassed RMs and the actual RMs - appearing as a distribution centred at zero (which does not look gaussian). This is the same for the annulus-bandpassed RMs, and is opposed to the annulus-convolved method, which does not appear to interact with the RM grid in a desirable manner.

\begin{figure}
    \includegraphics[width=\textwidth]{"Big_RM_histogram_2.png"}
    \centering
    \caption{Residual histogram plots of the actual POSSUM RMs compared to the corresponding various foreground removal methods: unaltered interpolation (Top left); crosshatch-bandpassed (Top right); annulus-bandpassed (Bottom left); and annulus-convolved (Bottom right).}
    \label{fig:big_hist}
\end{figure}


Figure \ref{fig:colour_maps_1} demonstrates the similarities between the crosshatch-bandpass and unaltered Hutschenreuter map, with them being related to each other in a linear manner, specifically with a gradient of approximately unity. This is ideal, as it means that the crosshatch-bandpass method is not deviating significantly from the Hutschenreuter map, only altering it subtly. The figure also demonstrates how scattered the RMs become near the Galactic midplane, hence it being plotted for colour to delineate between RMs near and far away from the midplane.

Figure \ref{fig:colour_maps_2} compares the unaltered Hutschenreuter map with the annulus-convolved method (a.k.a. the fixed-size annulus method). From the bottom two graphs, the annulus-bandpass is similar to the crosshatch bandpass, leaving most RMs as correlated.


Figure \ref{fig:colour_maps_3} compares the unaltered Hutschenreuter map with the annulus-convolved method (a.k.a. the fixed-sampling annulus method). There is a somewhat linear relationship between the actual RMs and the annulus-convolved RMs, ignoring the heavy scatter closer to the midplane. This gives credibility to this method, and the choice of having a small-sized annulus. From all three maps, it appears that there is not much of a correlation between the Hutschenreuter and POSSUM RMs. This implies a lot of smoothing is already performed by the Hutschenreuter map, as anticipated in section \ref{sec:intp}.


\begin{figure}
    \includegraphics[width=0.6\textwidth]{"RM_Colour_Maps.png"}
    \centering
    \caption{Three corner plots of all RM sampling points, describing the relationship between the several associated RM values and the Binned Maximum Absolute Galactic lattitude indicated by colour. Here, a comparison between POSSUM RMs, Hutschenreuter RMs, and Crosshatch-Bandpassed RMs is displayed.}
    \label{fig:colour_maps_1}
\end{figure}

\begin{figure}
    \includegraphics[width=0.6\textwidth]{"RM_Colour_Maps_Annulus.png"}
    \centering
    \caption{Three corner plots of all RM sampling points, describing the relationship between the several associated RM values and the Binned Maximum Absolute Galactic lattitude indicated by colour. Here, a comparison between POSSUM RMs, Hutschenreuter RMs, and Annulus-Bandpassed RMs is displayed.}
    \label{fig:colour_maps_2}
\end{figure}

\begin{figure}
    \includegraphics[width=0.6\textwidth]{"RM_Colour_Maps_Craig_Annulus.png"}
    \centering
    \caption{Three corner plots of all RM sampling points, describing the relationship between the several associated RM values and the Binned Maximum Absolute Galactic lattitude indicated by colour. Here, a comparison between POSSUM RMs, Hutschenreuter RMs, and Annulus-Convolved RMs is displayed.}
    \label{fig:colour_maps_3}
\end{figure}

%A corner plot of all RM sampling points, describing the relationship between the actual RMs, interpolated RMs, crosshatch-banpassed RMs, and Galactic lattitude. The Galactic lattitude is indicated by colour.

\section{Statistical Comparison of Foreground Removal Methods}
\label{sec:FR_stats}

\begin{figure}
    \includegraphics[width=10cm]{"t_dist_residuals.png"}
    \centering
    \caption{The sample of all RMs corrected with the Hutschenreuter Interpolation (Blue) and Crosshatch-Bandpassed Interpolation (Red). A bootstrap fitted t-distribution is drawn on the graph for each respective residual RM set.}
    \label{fig:t_dist}
\end{figure}

The corrected RM histograms for both bandpass methods displayed in figure \ref{fig:big_hist} do not follow a Gaussian distribution or a Cauchy distribution, but do follow a Student's t-distribution. This has been seen before in past analysis of residual RMs \citep{ID73}. By eye, it is quite clear that this is not the case for the annulus-convolved method, hence why no statistical analysis was conducted on it beyond an $R^2$ calculation. A chi-squared test can be used to compare the residual RM distributions with the t-distribution, with the t-distribution being fitted via a bootstrapping method. 1000 resamplings with replacement were used.

Table \ref{tab:fr_stats} displayed the resulting t-distribution fit and Pearson's chi-squared statistic, which tests for similarity instead of difference. The p-values for both were extremely small, despite the clear visual incongruency in the model and data set in figure \ref{fig:t_dist}. This is most likely caused by the t-distribution fitting very well with the tails and mid-section of the data histogram.

\begin{table}
    \centering
    \begin{tabular}{l l l l}
        \hline
        \multirow{2}{*}{\bfseries{Residual Set}} & \multicolumn{3}{l}{\bfseries{t-Distribution Fit}} \\
        & Centre & d.o.f. & Scale \\
        \hline
        Straight Interpolation & 1.461 $\pm$ 0.190 & 1.040 $\pm$ 0.0039 & 14.24 $\pm$ 0.0606 \\
        Crosshatch-Bandpassed & 1.648 $\pm$ 0.047 & 1.046 $\pm$ 0.0039 & 14.47 $\pm$ 0.0608 \\
        Annulus-Bandpassed & 1.617 $\pm$ 0.195 & 1.054 $\pm$ 0.0040 & 14.67 $\pm$ 0.0630 \\
        \hline
        & & & \\
    \end{tabular}
    \begin{tabular}{l l l}
        \hline
        \multirow{2}{*}{\bfseries{Residual Set}} & \multicolumn{2}{l}{\bfseries{$\chi^2$ Results}} \\
        & Statistic & p-value \\
        \hline
        Straight Interpolation & 0.01432 & 1.0442E-166 \\
        Crosshatch-Bandpassed & 0.01460 & 2.671E-166 \\
        Annulus-Bandpassed & 0.01385 & 2.105E-167 \\
        \hline
        & & \\
    \end{tabular}
    \begin{tabular}{l l}
        \hline
        \bfseries{Methods} & \bfseries{$R^2$ Statistic} \\
        \hline
        Interpolation vs. Crosshatch-Bandpassed & 0.9279 \\
        Interpolation vs. Annulus-Bandpassed & 0.8701 \\
        Interpolation vs. Annulus-Convolved & 0.003866 \\
        \hline
    \end{tabular}
    \caption{(Top) Student's t-distribution fit parameters for the residual histograms as displayed in figure \ref{fig:t_dist}. (Middle) Results of the $\chi^2$ test comparing residual histograms with the t-distrubution. (Bottom) $R^2$ statistics comparing the Hutschenreuter map with the other methods detailed in this section.}
    \label{tab:fr_stats}
\end{table}

The Pearson $R^2$ test can be performed in comparing the Hutschenreuter map with its corrected versions. Ideally, the $R^2$ statistic should be somewhat close to 1, as not a lot of sub-degree variations are expected (due to the single-degree scales of the Hutschenreuter map's sources). Table \ref{tab:fr_stats} presents the $R^2$ statistics for a series of compared cases.


\section{Other Methods}
\label{sec:other_methods}

There are other methods for foreground removal were not considered viable options or of enough importance to numerically analyse in this report. However, they may still offer useful methods for future researchers. The first is median filtering, which is like previously discussed convolutional blurring methods. \cite{ID39} provides a description of a fast median filtering algorithm for two-dimensional images. \cite{ID40} is also a reference describing the potential benefits of median filtering, including a more robust removal of noise due to the statistical properties of the median, and the preservation of edges. It was disregarded in this report due to its inherent non-linearity, leading to more computational complexity, and the debatable nature of whether this disadvantage is worth the advantages it can bring. Linear decomposition of line-of-sight RMs are also possible, attempted in \cite{ID21}. However, it is also quite mathematically complex and may not be conducive to a generalised algorithm, hence the lack of focus on this technique in this report. Despite the methods presented in the report, in section \ref{cha:derivation}, a non-altered Hutschenreuter map will be used with the justification found in section \ref{sec:los_dev}.


%%% Local Variables: 
%%% mode: latex
%%% TeX-master: "paper"
%%% End: 
