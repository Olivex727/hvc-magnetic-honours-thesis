\chapter{Introduction}
\label{cha:introduction}


\section{Thesis Statement}
\label{sec:thesisstatement}
I believe A is better than B.

\section{Introduction}
\label{sec:problemstatement}
Put your introduction here. You could use \textbackslash fix\{ABCDEFG.\} to
leave your comments, see the box at the left side. \ifix{You have to rewrite your
thesis!!!}

\section{Objectives, novelty, and expected outcomes}

The objective of this honours thesis is to:
\begin{enumerate}
\item Investigate the presence of magnetic fields surrounding circumgalactic HVCs using RMs collected from the POSSUM survey
\item Determine if these magnetic fields, from a “rough estimation” (i.e. is the number big enough), can allow HVCs to travel through a halo relatively undisturbed
\item (if there is time) More in-depth modelling surrounding the specific structure of magnetic fields surrounding HVCs and if this structure is preventing the formation of Kelvin-Helmholtz instabilities
\end{enumerate}

While previous reports exist that have measured magnetic fields surrounding circumgalactic HVCs, most of this data is from HVCs in the Smith Cloud or Leading Arm. There so far has not been a substantial analysis of magnetic fields throughout HVCs across the entire southern hemisphere. This is important, as it could be the case that the Smith Cloud and Leading Arm are exceptional compared to the relatively homogenous Galactic halo.

Novelty also comes in the scope of the analysis, as a broad measuring of HVCs in the southern hemisphere means a much larger data set for future researchers to pull on. This report takes advantage of the fact that only recently has the POSSUM survey catalogued enough data to perform this operation.

Of course, the report will rely heavily on previous data. Magnetic field measurements in Smith Cloud and Leading Arm HVCs, simply as a means of providing a test data set, from which it can be confirmed that the program is working as intended. Previous location and spectral analysis of HVCs, specifically in the measurement of HI emission and absorption spectra, is used to not only know where the HVCs are, but also to determine important factors like optical depth or internal ionisation.

Because this report is one of the first to measure magnetic fields surrounding HVCs, there are a few potential outcomes that this report needs to anticipate:
\begin{itemize}
\item There are magnetic fields that are strong enough to support HVCs. This would mean that HVCs can act as a significant explanation for how galaxies take in external gas. Future research from this would involve more in-depth modelling on magnetic fields, the setup of a northern hemisphere equivalent to POSSUM, investigation of extragalactic HVCs, or determining if the gas content in HVCs can fully account for the fuel demands of star-forming galaxies.
\item Magnetic fields exist surrounding HVCs in the halo, however they are only strong enough to partially explain HVC transport, meaning that this report would give a percentage estimation to how much is missing in the explanatory power of collected magnetic fields. Future research would involve investigating other possible sources of stability for HVCs or attempting to find a different method for how gas enters galactic disks.
\item The magnetic fields are either non-existent or too weak to support HVCs i.e. a null result. This potentially eliminates HVCs as an explanation for gas infall, meaning that we are back at square one. Future research would involve checking to confirm if the null result of the report is correct, or if there is wiggle room in the data for other HVCs in the Milky Way. It may also involve completely abandoning HVCs as a candidate for gas infall after more careful analysis is done.
\end{itemize}