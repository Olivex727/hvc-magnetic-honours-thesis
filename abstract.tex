\chapter*{Abstract}
\addcontentsline{toc}{chapter}{Abstract}
%\vspace{-1em}

\noindent\textit{Context.} High Velocity Clouds (HVCs) are a proposed solution to how extragalactic gas enters into star-forming galaxies. However, the presence of enveloping magnetic fields is required to ensure that HVCs can travel through the halo without being torn apart by ram pressure. This hypothesis, in the context of HVCs, is referred to as 'magnetic draping'.

\noindent\textit{Aims.} This report aims to detect the signature of magnetic draping in HVCs and evaluate the strength of magnetic fields enveloping HVCs - both using a newly constructed rudimentary algorithm applicable to a broad range of HVCs. A secondary goal of the report is on the investigation of image processing of interpolated data to solve anticipated resolution issues with higher-quality SKA-era interpolated rotation measure data.

\noindent\textit{Data.} Recent rotation measure data was retrieved in May 2024 from the Polarisation Sky Survey of the Universe's Magnetism (POSSUM) and HVC data was retrieved the Galactic All Sky Survey (GASS). An interpolation of legacy rotation measure data was retrieved from the Galactic Faraday rotation sky 2020.

\noindent\textit{Methods.} The Kolmogorov-Smirnov test assuming a spherical HVC morphology was used to detect magnetic draping by determining the difference between in-HVC and out-HVC rotation measure populations. Two new statistical methods to evaluate the magnitude of magnetic draping were proposed, in addition to the use of a previous method sanctioned by literature. Several image processing techniques, including Fourier analysis, were visually and statistically analysed as to their efficacy at removing objects of a particular angular scale.

\noindent\textit{Results.} Of 13 HVCs analysed, a magnetic field signature of 11 HVCs was found to within 99.9\% confidence. One of the proposed methods, called 'variance subtraction' displayed statistical agreement with the traditional weighted mean method. The average HVC magnetic field is an order of magnitude greater than or equal to than in simulations, at scales of 1 \textmu G. The results of the derivation should be taken with great caution, however, due to large uncertainties in resultant magnetic field estimates. A two-dimensional bandpass method was found to be adequate at preserving the foreground, with potential for it to correct out objects of interest.

\newpage 

\ % The empty page

\newpage


%%% Local Variables: 
%%% mode: latex
%%% TeX-master: "paper"
%%% End: 