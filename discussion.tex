\chapter{Discussion}
\label{cha:discussion}

There have been several caveats made in the explanation of the data collection, methodology, results, and analysis of results. The goal of this section is to summarize such information (involving a re-invoking of certain statements made in previous sections) and to build on that. Like the rest of the report structure, this section is split under the two different research questions investigated in this thesis, starting with the secondary goal of the report and finishing with the primary goals of the report.

\section{Foreground Removal}
\label{sec:fr_disc}

A major shortfall of the foreground subtraction investigation is the lack of statistical analysis, which will be a common theme throughout this section. In some cases, this is justified, as statistical analysis does not need to be performed especially when the method visually appears as adequate. At the same time, this investigation is limited from a lack of characterisation of the true RM sky, including simulations to predict the distribution of corrected RMs and high-quality data to interpolate with.


This report is not meant to act as “closing the book” on research into foreground subtraction techniques, but to point out important considerations to make as research moves into the SKA era, and to conduct a preliminary analysis of how to tackle these predicted problems. This comes in the explanation in section [SECREF], which highlights how higher-definition data will inevitably result in issues involving foreground correction for objects of a particular angular size. Thus, the need to extend beyond interpolation techniques.


The NUFFT method is one example of a method that can be assessed visually, and without the need for statistical analysis. Figure [FIG] clearly shows that even carrying out an interpolation using a NUFFT requires a large sample size to be adequate. It would not be appropriate to completely write-off NUFFTs as an interpolation and bandpass technique, as with a more complete POSSUM dataset, and future SKA RM projects, millions of RM points would be adequate to interpolate and bandpass using a NUFFT. There will simply not be enough data in the few years after this paper.


Given that the Hutschenreuter map interpolates data that is at a spatial sampling frequency of 1 sampling point per degree squared, attempting to remove objects on the scale of 1 degree should not alter the interpolation at all. However, there are several factors that prevents this from being an accurate description of the Hutschenreuter map: H-alpha inclusion providing more detailed information; more RM sources being detected by telescopes close to the galactic midplane; and distortion effects occurring on the edges of the cartesian map. The latter two can be accounted for by performing the filtering on a reduced form of the dataset that does not include extremes in galactic latitude. However, this was not done with the statistical analysis of foreground removal techniques, as it's important to determine the effectiveness of a given method at accounting for these changes.


This is the justification for the use of a Pearson's R-squared statistic, as it can determine how similar the pre and post bandpass and convolution methods are at preserving the broader RM sky. With the expectation being that the R-squared statistic is close to unity but not necessarily exactly unity. One major downside is that there is no way to quantify the appropriate R-squared statistic that should be desired. Thus, only a high R-squared statistic is possible qualifier.


It is clear from the R-squared statistic that, even with a less-exact method of determining validity, only the Crosshatch-Bandpass method is viable as an interpolation correction method. It can be pointed out that it may be unfair to compare the Annulus-Convolved method to the raw interpolation. However, it is justified in the sense that the raw interpolation has been used as the primary method in past literature, and from figure [FIG], it is demonstrably useful at removing the anisotropy produced by the ISM.


Both annulus methods had major points of failure regarding the specific technique that they relied on to correct the RMs. When the RM sky is meant to be full of objects and obstructions of different sizes and overlapping positions, it can make it difficult to use a fixed size or RM count to correct. The RM distributions in figure [FIG] show visually that a method involving the averages of a set of RMs, and corrections based on that average are possible, but the effective 'goldilocks zone' for a good size or sampling count is small and would require large simulations to find the ideal size – even then, it may not be adequate for certain objects.


The Crosshatch-Bandpass method seems to be the most appropriate as a first step in SKA-era foreground interpolation. However, with the rippling effects shown in figure [FIG], there are improvements that can be made. The Fourier Transform of a top-hat function is inevitably going to introduce rippling. This may not be as much of the case with the Tukey or Gaussian windows, which is much more common in other signal processing analyses. The choice to reduce bandpass opacity is another option that clearly removes ripples, at the cost of dampening the intended goal of using FFT signal processing methods. Some combination of these may be useful for future research relying on RM-correction.


The interesting result provided in [SOURCE] clearly still shows with a repeated analysis on the Hutschenreuter map, and the improved Crosshatch-Bandpass method. There is still little reason for why corrected RMs take this shape, or if this is meant to be the true distribution of RMs post ISM correction. It is rather crucial that research be conducted into what the true distribution of ISM-corrected RMs should be. It is trivial to understand why the distribution should be symmetrical and centred at zero, as extragalactic polarisation is effectively random and not biased based on sign. The Student's t-distribution is one of many different distributions satisfying these properties that are viable candidates for representing corrected RMs, it needs to be explained why this specific distribution is the one that RMs follow.


Under the assumption that it is already known why and expected that the t-distribution is a correct reflection of reality, both the original and the Crosshatch-Bandpass filtered Hutschenreuter map align with the expected t-distribution incredibly well, with the parameters for the t-distribution fit being similarly identical. As mentioned in section [SECREF], despite the clear visual incongruence, the important property of the t-distribution which appeared to not work with other distributions centred at zero, is the tailed prevalence. This ultimately gave such a small p-value in the Pearson's chi-squared test. Both distributions additionally are centred close to zero, with the systematic error being most likely a consequence of measurement bias or a consequence of the POSSUM dataset covering only a specific proportion of the sky. The latter can be seen in how the ISM's anisotropic distribution doesn't seem exactly symmetric in figure [FIG], connecting it with the amount of RMs recorded by POSSUM in a specific quadrant of the southern hemisphere, seen in figure [FIG].


As mentioned, the results in this report regarding foreground techniques are results which may not immediately show utility in the next year or two. However, with more accurate interpolation techniques, like the one seen in [SOURCE], the pre-SKA and SKA-era analyses of the RM sky will greatly benefit from the preliminary research direction opened by this report.

\section{Magnetic Field Derivation}
\label{sec:mag_disc}

\begin{itemize}
    \item Data Collection
    \begin{itemize}
        \item HI and H-alpha Data (Not needed!)
    \end{itemize}
    \item Neccesity of Assumptions
    \item Validity of Derivation Methods and VGSR
    \item Uncertainty Analysis
\end{itemize}

\subsection{Data Collection}
\label{ssec:B1}

E

\subsection{Neccesity of Assumptions}
\label{ssec:B2}

E

\subsection{Validity of Detection Methods}
\label{ssec:B5}

E

\subsection{Validity of Derivation Methods}
\label{ssec:B3}

E

\subsection{Uncertainty Analysis}
\label{ssec:B4}

E
